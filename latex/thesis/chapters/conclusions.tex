\section{Conclusions}
In this master thesis, we developed a novel approach to automatically generate tags for OpenML dataset descriptions using topic modeling techniques. Our research addressed four main research questions, making several key contributions to both the OpenML platform and the field of topic modeling.

Addressing \hyperref[rq1]{\textbf{RQ1}} (What are the specifications of the OpenML dataset descriptions, and what impact may they have on model performance?), our exploratory data analysis revealed that most descriptions are relatively short and vary considerably in readability and complexity. We found that many descriptions contain domain-specific technical terms and that a significant portion already have associated tags. This understanding guided our data augmentation strategy, where we enhanced descriptions with metadata, features, and information from original sources, improving the information content available for topic modeling.

For \hyperref[rq2]{\textbf{RQ2}} (What are the different approaches to topic modeling that can be applied to the OpenML dataset descriptions, and what are the tradeoffs involved in their use?), we developed a modular pipeline that extends the BERTopic model. After evaluating various approaches, we found that combining BERTopic with advanced language models and zeroshot classification offered the best balance between computational efficiency and tag quality. The modular nature of our pipeline ensures it can evolve alongside advances in language models and embedding techniques, addressing the need for long-term sustainability.

In response to \hyperref[rq3]{\textbf{RQ3}} (What are suitable automated evaluation metrics for assessing the quality of the topics and terms extracted by the topic model?), we implemented and analyzed multiple evaluation metrics including NPMI, diversity, and silhouette scores. Our statistical analysis showed that our model significantly outperformed baseline approaches including LDA, NMF, and Top2Vec. Through Bayesian optimization of our base BERTopic model, we were able to reach a good performance across these metrics while maintaining computational efficiency.

Finally, addressing \hyperref[rq4]{\textbf{RQ4}} (What are suitable human evaluation methods for assessing the quality of the topics and terms extracted by the topic model?), we designed and conducted a comprehensive human evaluation study with 21 participants. Our evaluation framework assessed relevance, generality, coverage, and shared coverage of tags, providing insights that automated metrics alone could not capture. The results showed that our model generates tags that are significantly more relevant and provide better coverage compared to the baseline approach, while maintaining a good balance between specific and general tags, and are close in performance to human-generated tags. We complemented this with a large-scale automated evaluation using \textit{GPT-4-mini}, which validated our findings across the entire OpenML dataset. The model demonstrated robust performance with high accuracy in intruder detection tasks and consistently high scores for relevance, generality, and coverage across different types of tags.

From a practical perspective, our approach provides OpenML with an automated, scalable solution for dataset tagging that produces high-quality, human-readable tags. The modular nature of our pipeline ensures that it can evolve alongside advances in language models and embedding techniques, making it a sustainable long-term solution for the platform.

This work has implications beyond OpenML, as the techniques we developed could be applied to similar problems in other domains where automated categorization of technical or scientific content is needed. Our approach to combining traditional topic modeling with modern language models and zeroshot classification represents a novel contribution to the field of topic modeling itself.